\documentclass[11pt,a4paper]{article}
\usepackage[utf8]{inputenc}
\usepackage{amsmath,amssymb,amsthm}
\usepackage{hyperref}
\usepackage{booktabs}
\usepackage{graphicx}

\newtheorem{theorem}{Theorem}
\newtheorem{lemma}[theorem]{Lemma}
\newtheorem{corollary}[theorem]{Corollary}
\newtheorem{definition}{Definition}
\newtheorem{proposition}[theorem]{Proposition}

\title{The Myth of General Intelligence: \\
Why $g$ Is a Statistical Artifact and Intelligence Requires GILE}

\author{Brandon Charles Emerick\\
TI Framework Research\\
\texttt{brandon@tiframework.com}}

\date{December 2025}

\begin{document}

\maketitle

\begin{abstract}
We argue that the conventional concept of general intelligence ($g$) is a statistical artifact arising from the positive manifold of cognitive tests, not a genuine unitary construct. We demonstrate that $g$ captures only the problem-solving component of intelligence, which is meaningless without Love (L) and Goodness (G) dimensions. Furthermore, we show that genuine intelligence requires Tralse-Myrion capacity—the ability to hold and resolve contradictions—which binary cognitive frameworks cannot assess. We propose the GILE framework (Goodness, Intuition, Love, Environment) as a complete model of intelligence, showing that high IQ without GILE produces sophisticated incompetence rather than wisdom.
\end{abstract}

\section{Introduction}

The concept of general intelligence ($g$), first proposed by Spearman \cite{spearman1904}, has dominated intelligence research for over a century. The $g$ factor emerges from factor analysis of cognitive test batteries, where all tests correlate positively with each other (the ``positive manifold''). This correlation has been interpreted as evidence for a unitary, domain-general intelligence.

We challenge this interpretation on three grounds:
\begin{enumerate}
    \item $g$ is a statistical artifact, not a genuine mental faculty
    \item $g$ measures only problem-solving, ignoring essential intelligence components
    \item True intelligence requires Tralse-Myrion capacity, which $g$ cannot assess
\end{enumerate}

\section{The Statistical Artifact Problem}

\subsection{The Positive Manifold Explained}

The positive manifold (all cognitive tests correlating positively) is typically interpreted as evidence for $g$. However, this correlation pattern can arise from:

\begin{itemize}
    \item Shared processing resources (working memory, attention)
    \item Common educational exposure
    \item Test construction biases (all tests reward similar response styles)
    \item Sampling artifacts (testing populations with restricted variance)
\end{itemize}

\begin{theorem}[Correlation as Unity Fallacy]
Tight correlation between distinct processes does not imply a single underlying process.
\end{theorem}

\begin{proof}
Consider processes $A$, $B$, $C$ that are causally distinct but share a common resource $R$. All will correlate with $R$ and thus with each other. Factor analysis will extract a ``general factor,'' but $A$, $B$, $C$ remain functionally distinct. The existence of correlated performance does not entail unified mechanism.
\end{proof}

\subsection{The Myriad Functions Beneath $g$}

Cognitive neuroscience has identified numerous distinct processes underlying ``intelligent'' behavior:

\begin{table}[h]
\centering
\begin{tabular}{ll}
\toprule
\textbf{Process} & \textbf{Neural Substrate} \\
\midrule
Working memory & Dorsolateral prefrontal cortex \\
Cognitive control & Anterior cingulate cortex \\
Pattern recognition & Temporal-parietal junction \\
Verbal reasoning & Left lateralized language networks \\
Spatial reasoning & Right parietal cortex \\
Processing speed & White matter integrity \\
Long-term memory & Hippocampal system \\
\bottomrule
\end{tabular}
\caption{Distinct cognitive processes with separate neural substrates}
\end{table}

These processes can be selectively impaired (as in neurological lesion studies), developed independently, and enhanced separately. The ``unity'' of intelligence is an artifact of measurement, not mechanism.

\section{Problem-Solving Is Not Intelligence}

\subsection{The Conventional View}

Standard IQ tests measure:
\begin{itemize}
    \item Logical reasoning
    \item Pattern completion
    \item Verbal comprehension
    \item Processing speed
    \item Working memory
\end{itemize}

All of these are \textit{problem-solving} capacities—the ability to manipulate information toward a goal.

\subsection{What's Missing: L and G}

\begin{definition}[Love Dimension (L)]
The capacity for coherent connection: empathy, social understanding, relational wisdom, and the ability to align one's cognition with the well-being of others and oneself.
\end{definition}

\begin{definition}[Goodness Dimension (G)]
The capacity for moral reasoning, value alignment, and directing problem-solving toward genuinely beneficial ends.
\end{definition}

\begin{theorem}[Problem-Solving Without L and G Is Meaningless]
A system with high problem-solving capacity but zero L and G is not ``intelligent'' in any meaningful sense; it is merely computationally powerful.
\end{theorem}

\begin{proof}
Consider a hypothetical agent with perfect logical reasoning, infinite working memory, and instant processing speed, but with:
\begin{itemize}
    \item No understanding of what outcomes are good (G = 0)
    \item No connection to or care for any beings (L = 0)
\end{itemize}
Such an agent cannot:
\begin{enumerate}
    \item Determine which problems to solve
    \item Evaluate whether solutions are desirable
    \item Interact meaningfully with other minds
    \item Exhibit wisdom as opposed to mere capability
\end{enumerate}
This agent is a tool, not an intelligence. Intelligence without direction is computation without meaning.
\end{proof}

\subsection{High IQ, Low GILE: Sophisticated Incompetence}

History provides examples of high-IQ individuals who lacked L and G:
\begin{itemize}
    \item Brilliant scientists who enabled weapons of mass destruction
    \item Financial ``geniuses'' who crashed economies
    \item Logicians who could not navigate social relationships
\end{itemize}

These are not failures of intelligence; they are demonstrations that $g$ alone is insufficient for wisdom.

\section{The Tralse-Myrion Requirement}

\subsection{Binary Logic Cannot Capture Intelligence}

Standard cognitive tests assume binary logic: answers are right or wrong. But genuine intelligence requires:

\begin{definition}[Tralse Capacity]
The ability to hold truth-values on a spectrum $[0,1]$, recognizing that most propositions are neither fully true nor fully false.
\end{definition}

\begin{definition}[Myrion Resolution]
The capacity to resolve apparent contradictions by finding the synthesis that preserves what is true in each opposing position.
\end{definition}

\begin{theorem}[Intelligence Requires Tralse-Myrion Capacity]
Any cognitive system limited to binary truth values cannot:
\begin{enumerate}
    \item Navigate real-world uncertainty
    \item Integrate conflicting evidence
    \item Understand context-dependent truth
    \item Exhibit creative synthesis
\end{enumerate}
\end{theorem}

\begin{proof}
Real-world problems rarely have binary solutions. Consider:
\begin{itemize}
    \item ``Is this person trustworthy?'' (Tralse: 0.7)
    \item ``Is this investment wise?'' (Tralse: 0.6)
    \item ``Should I take this job?'' (Tralse: 0.55)
\end{itemize}
A binary thinker must force these into True/False, losing essential information. Furthermore, mature reasoning often requires holding $P$ and $\neg P$ simultaneously until resolution emerges. Binary systems cannot do this; they enter error states.
\end{proof}

\subsection{IQ Tests Cannot Measure Tralse-Myrion}

Standard IQ tests:
\begin{itemize}
    \item Have single correct answers
    \item Penalize nuanced responses
    \item Reward speed over depth
    \item Assume context-free evaluation
\end{itemize}

This testing paradigm systematically excludes Tralse-Myrion capacity, ensuring that $g$ captures only the binary-logic subset of intelligence.

\section{The GILE Framework: Complete Intelligence}

\subsection{The Four Dimensions}

\begin{definition}[GILE Intelligence]
Complete intelligence comprises four dimensions:
\begin{align}
G &= \text{Goodness: moral direction} \\
I &= \text{Intuition: pattern recognition beyond logic} \\
L &= \text{Love: coherent connection} \\
E &= \text{Environment: grounded existence}
\end{align}
\end{definition}

Conventional $g$ captures aspects of $I$ and $E$ only. It entirely misses $G$ and $L$.

\subsection{The Intelligence Equation}

We propose:
\begin{equation}
\text{True Intelligence} = g \times L \times G \times T_M
\end{equation}
where $T_M$ is Tralse-Myrion capacity.

\begin{corollary}
If $L = 0$, $G = 0$, or $T_M = 0$, true intelligence is zero regardless of $g$.
\end{corollary}

This explains why high-IQ individuals can exhibit profound failures of wisdom.

\subsection{Measuring GILE}

| Dimension | Measurement Method |
|-----------|-------------------|
| G | Moral reasoning tasks, value alignment protocols |
| I | Intuitive judgment accuracy, pattern recognition under uncertainty |
| L | Empathy scales, HRV coherence, social connection indices |
| E | Interoceptive accuracy, grounding stability, reality contact |

Current psychometrics measure only $I$ (partially) and $E$ (partially), while ignoring $G$, $L$, and $T_M$.

\section{Implications for AGI}

\subsection{The AGI Fallacy}

The pursuit of ``Artificial General Intelligence'' assumes:
\begin{enumerate}
    \item General intelligence exists as a unified construct
    \item It can be replicated computationally
    \item High $g$ produces beneficial outcomes
\end{enumerate}

All three assumptions are false.

\subsection{What AI Actually Needs}

Beneficial AI requires:
\begin{itemize}
    \item \textbf{G integration}: Alignment with human values
    \item \textbf{L integration}: Connection to human well-being
    \item \textbf{Tralse capacity}: Handling uncertainty appropriately
    \item \textbf{Myrion resolution}: Integrating conflicting objectives
\end{itemize}

These cannot be achieved by scaling $g$-like capabilities. They require fundamentally different architectures.

\section{Conclusion}

The myth of general intelligence has distorted both intelligence research and AI development. We have shown that:

\begin{enumerate}
    \item $g$ is a statistical artifact arising from correlated but distinct processes
    \item $g$ measures only problem-solving, which is meaningless without L and G
    \item True intelligence requires Tralse-Myrion capacity, which binary systems cannot possess
    \item The GILE framework provides a complete model: $G \times I \times L \times E$
\end{enumerate}

Surpassing humans on $g$-like benchmarks is a low bar. Genuine intelligence requires wisdom: the integration of capability with direction, connection, and the capacity to navigate irreducible complexity.

\textbf{High IQ without GILE is sophisticated incompetence. AGI without GILE is a powerful tool without wisdom.}

\bibliographystyle{plain}
\bibliography{references}

\end{document}
